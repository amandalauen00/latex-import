% ===============================================
% MATH 373: Intro to Numerical Analysis           Fall 2021
% program_5.tex
% June 11, 2021
% ===============================================

\documentclass{article}

% load packages
\usepackage{amsmath,amsfonts,graphicx,amsthm,amssymb,hyperref,xcolor}

% Define default environments
\newenvironment{theorem}[2][Theorem]{\begin{trivlist}
\item[\hskip \labelsep {\bfseries #1}\hskip \labelsep {\bfseries #2.}]}{\end{trivlist}}
\newenvironment{lemma}[2][Lemma]{\begin{trivlist}
\item[\hskip \labelsep {\bfseries #1}\hskip \labelsep {\bfseries #2.}]}{\end{trivlist}}
\newenvironment{claim}[2][Claim]{\begin{trivlist}
\item[\hskip \labelsep {\bfseries #1}\hskip \labelsep {\bfseries #2.}]}{\end{trivlist}}
\newenvironment{problem}[2][Problem]{\begin{trivlist}
\item[\hskip \labelsep {\bfseries #1}\hskip \labelsep {\bfseries #2.}]}{\end{trivlist}}
\newenvironment{proposition}[2][Proposition]{\begin{trivlist}
\item[\hskip \labelsep {\bfseries #1}\hskip \labelsep {\bfseries #2.}]}{\end{trivlist}}
\newenvironment{corollary}[2][Corollary]{\begin{trivlist}
\item[\hskip \labelsep {\bfseries #1}\hskip \labelsep {\bfseries #2.}]}{\end{trivlist}}

\newenvironment{solution}{\begin{proof}[Solution]}{\end{proof}}

%
% Convention for citations is authors' initials followed by the year.
% For example, to cite a paper by Leighton and Maggs you would type
% \cite{LM89}, and to cite a paper by Strassen you would type \cite{S69}.
% (To avoid bibliography problems, for now we redefine the \cite command.)
% Also commands that create a suitable format for the reference list.
\renewcommand{\cite}[1]{[#1]}

\def\beginrefs{\begin{list}%
        {[\arabic{equation}]}{\usecounter{equation}
         \setlength{\leftmargin}{2.0truecm}\setlength{\labelsep}{0.4truecm}%
         \setlength{\labelwidth}{1.6truecm}}}
\def\endrefs{\end{list}}
\def\bibentry#1{\item[\hbox{[#1]}]}


% Define Shortcuts
\def\ds{\displaystyle}

\begin{document}



% ------------------------------------------ %
%                 START HERE             %
% ------------------------------------------ %

\large

{\Large Math 373, Introduction to Numerical Analysis}


{\Large Program 5} \par \medskip \noindent
%
{\bf Summary:} The subject of this assignment is a variation of a basic problem from differential equations. Recall the mass spring system (the harmonic oscillator):
\begin{equation}
m\frac {d^2x}{dt^2} + c \frac {dx}{dt} + k x = 0,
\label{e:harmonic_oscillator}
\end{equation}
where $x(t)$ is the position of the mass at time $t$, $m$ is the mass, $c$ is the friction coefficient, and $k$ is the spring constant. The problem posed in (\ref{e:harmonic_oscillator}) is second order and linear, which can be solved analytically. However, we are going to add a detail that $c$ varies with time. The problem you are to solve is one where $m = 20$ kg, $k = 20$ N/m, and $c$ is given by
$$c = \begin{cases}
4 & t<0 \\
4 - 0.02\ t^2 & 0 \le t \le 10 \\
2 & t > 10.
\end{cases}
$$
\par \bigskip \par \noindent
In this assignment, you are to construct a MATLAB program that takes the input of the time for the model to run in seconds, the initial position of the mass, and the initial velocity of the mass. 

Your program will specifically have the following format: \par \medskip
function [flag, T, W] = proge\#\#\#\#\#\#(Tf,x0,xp0), \par \medskip \noindent
%
where the file name is proge followed by your six digit Math 373 student number  followed by ".m" to make it an m-file. The input of Tf is the length of time to run the model in (\ref{e:harmonic_oscillator}) using the unit of measure of seconds along with the initial starting time set as t=0. The input of $x0$ is the initial starting position of the mass using the unit of measure of meters and the $xp0$ is the initial velocity of the mass using the units of meters per second. This assignment also specifies the of the use of the ode23 solver in MATLAB to solve the initial value problem. 

The output flag variable indicates a successful run, or a run that has some kind of error. If the program runs without an issue then flag = 0. You can assume the user will provide numeric input for each variable. Your program should check that the input values are feasible to generate a solution, but if the input values are not feasible then flag = 1 along with T and W both being equal to -9. If the input does generate a feasible solution then the output of T and W are the arrays that are generated by ode23. Remember that W would have two columns with one column representing the values of position for every time step and the other column representing the velocity of the mass at every time step.  

Please include a discussion in your report on how you validated that your program is producing an accurate calculation. What behavior should you expect for this problem. What are the units of measure for $c$ in order for the units  to correctly align for the model presented (\ref{e:harmonic_oscillator}) and the stated units for the other variables?  If the user input results in a feasible solution then your program should also plot the resulting solution that shows a plot of displacement and velocity in the same figure for each time step. Your plot should include sufficient labeling. Please note your program should only return the output variables along with the plot (when appropriate) and should not output or print anything else. 

\par \bigskip \par

% Add references here, list alphabetically according to last name of primary author.
\section*{References}
\beginrefs

\bibentry{LB16}{\sc Leon Brin},
{\it Tea Time Numerical Analysis (Experiences in Mathematics)  (2nd ed.)}, 2016. Website: \href{http://lqbrin.github.io/tea-time-numerical/}{lqbrin.github.io/tea-time-numerical} .

\bibentry{KK09} {\sc Autar Kaw} and {\sc E. Eric Kalu}, {\it Numerical Methods with Applications (2nd ed.)}, 2009. Website: \href{http://autarkaw.com/books/numericalmethods/index.html}{book website} .

\bibentry{KR21} {\sc Kyle Riley}, Class Lecture, Math 373: Introduction to Numerical Analysis, Lecture, August 2021. 
\endrefs
\par \bigskip \noindent


\end{document}