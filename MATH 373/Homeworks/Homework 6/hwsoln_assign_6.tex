% ===============================================
% MATH 373: Intro to Numerical Analysis           Fall 2021
% hwsoln_assign_6.tex
% July 1, 2021
% ===============================================

\documentclass{article}

% load packages
\usepackage{amsmath,amsfonts,graphicx,amsthm,amssymb,hyperref,xcolor,enumitem}

% Define default environments
\newenvironment{theorem}[2][Theorem]{\begin{trivlist}
\item[\hskip \labelsep {\bfseries #1}\hskip \labelsep {\bfseries #2.}]}{\end{trivlist}}
\newenvironment{lemma}[2][Lemma]{\begin{trivlist}
\item[\hskip \labelsep {\bfseries #1}\hskip \labelsep {\bfseries #2.}]}{\end{trivlist}}
\newenvironment{claim}[2][Claim]{\begin{trivlist}
\item[\hskip \labelsep {\bfseries #1}\hskip \labelsep {\bfseries #2.}]}{\end{trivlist}}
\newenvironment{problem}[2][Problem]{\begin{trivlist}
\item[\hskip \labelsep {\bfseries #1}\hskip \labelsep {\bfseries #2.}]}{\end{trivlist}}
\newenvironment{proposition}[2][Proposition]{\begin{trivlist}
\item[\hskip \labelsep {\bfseries #1}\hskip \labelsep {\bfseries #2.}]}{\end{trivlist}}
\newenvironment{corollary}[2][Corollary]{\begin{trivlist}
\item[\hskip \labelsep {\bfseries #1}\hskip \labelsep {\bfseries #2.}]}{\end{trivlist}}

\newenvironment{solution}{\begin{proof}[Solution]}{\end{proof}}

%
% Convention for citations is authors' initials followed by the year.
% For example, to cite a paper by Leighton and Maggs you would type
% \cite{LM89}, and to cite a paper by Strassen you would type \cite{S69}.
% (To avoid bibliography problems, for now we redefine the \cite command.)
% Also commands that create a suitable format for the reference list.
\renewcommand{\cite}[1]{[#1]}

\def\beginrefs{\begin{list}%
        {[\arabic{equation}]}{\usecounter{equation}
         \setlength{\leftmargin}{2.0truecm}\setlength{\labelsep}{0.4truecm}%
         \setlength{\labelwidth}{1.6truecm}}}
\def\endrefs{\end{list}}
\def\bibentry#1{\item[\hbox{[#1]}]}


% Define Shortcuts
\def\ds{\displaystyle}

\begin{document}



% ------------------------------------------ %
%                 START HERE             %
% ------------------------------------------ %

\large

{\Large Math 373, Introduction to Numerical Analysis}


{\Large Homework Assignment 6} 
 \medskip \par

% -----------------------------------------------------
{\bf Section 1 Boundary Value Problems}  \\
%
Question 1.A: Outline the key components of the shooting method. What changes can be made to help improve the accuracy of a shooting method solver? 
\par \medskip \noindent 
%
Question 1.B: Outline the key components of the finite differences method. What changes can be made to help improve the accuracy of the finite differences method? 
\par \medskip \noindent 
%
Question 1.C: Consider the following boundary value problems and identify if the shooting method and/or the finite differences method can be used to solve the given problem.  If neither method can be used then please state neither and also explain your answer. 

\begin{itemize}
\item $\ds y'' + y' - 2y = t^2; \ \ y(0)=4 \ {\rm and}\ \ y(3)=-1$ {\color{teal} both}
\item $\ds y'' + y' - 2y = t^2; \ \ y'(0)=1 \ {\rm and}\ \ y'(3)=-2$ {\color{teal} finite differences}
\item $\ds y'' + y'y = \sin (t); \ \ y(0)=3 \ {\rm and}\ \ y(4)= 1$ {\color{teal} shooting}
\item $\ds y'' + y'y = \sin (t); \ \ y(0)=3 \ {\rm and}\ \ y'(4)= 1$ {\color{teal} neither (could adapt shooting)}
\item $\ds y'' + y'y = \sin (t); \ \ y'(0)=3 \ {\rm and}\ \ y'(4)= 1$ {\color{teal} neither}
\item $\ds y'' + 2 y' + 6 y = e^t; \ \ y(1)=2 \ {\rm and}\ \ y(5)= -1$ {\color{teal} both}
\item $\ds y''' + 2 y'' + 6 y = e^t; \ \ y'(0)=2 \ {\rm and}\ \ y(5)= -1$ {\color{teal} neither}
\item $\ds y'' + 2 y'y = e^{-t} \ \ y'(0)=2 \ {\rm and}\ \ y'(1)= -1$ {\color{teal} neither}
\item $\ds y'' + 2 y'y = e^{-t} \ \ y(0)=2 \ {\rm and}\ \ y(1)= -1$ {\color{teal} shooting}
\end{itemize}
\par \medskip \noindent 
%
Question 1.D: Use finite difference using the centered difference approximation for each derivative for the following BVP
$$y'' + 2y' + 6y = t^2 + 1, \ \ y'(0)=1, {\rm and} \ \ y'(2) = 2.$$
\par \noindent
{\color{teal}$$ \begin{pmatrix} (6h^2-h-1) & (1+h) &  &  0 \cr (1-h) & (6h^2-2) & (1+h) & & & \cr  \ \ \ \ \ \  \ddots & \ddots & \ \ \ \ \ \ \ddots &  & \cr  \cr & (1-h) & (6h^2-2) &  (1+h) \cr 0 &  & (1-h) & (6h^2+h-1) \end{pmatrix} \begin{pmatrix} y_1 \cr y_2 \cr \vdots \cr \vdots \cr y_{n-2} \cr y_{n-1}\end{pmatrix} $$
$$= \begin{pmatrix} h^2 (t_1^2+1)+h(1-h) \cr h^2(t_2^2+1) \cr \vdots \cr \vdots \cr h^2(t_{n-2}^2+1) \cr h^2(t_{n-1}^2+1)-2h(1+h)\end{pmatrix}.$$}
\par \medskip \noindent 
%
Question 1.E:  Use finite difference using the centered difference approximation for the second derivative and the forward difference for the first derivative for the following BVP
$$y'' - 3y' + 6y = t^2 + 1, \ \ y(0)=3, {\rm and} \ \ y(1) = 5.$$
\par \noindent
{\color{teal}$$ \begin{pmatrix} (6h^2+3h-2) & (1-3h) &  &  0 \cr 1 & (6h^2+3h-2) & (1-3h) & & & \cr  \ \ \ \ \ \  \ddots & \ddots & \ \ \ \ \ \ \ddots &  & \cr  \cr & 1 & (6h^2+3h-2) &  (1-3h) \cr 0 &  & 1 & (6h^2+3h-2) \end{pmatrix} \begin{pmatrix} y_1 \cr y_2 \cr \vdots \cr \vdots \cr y_{n-2} \cr y_{n-1}\end{pmatrix} $$
$$= \begin{pmatrix} h^2 (t_1^2+1)-3 \cr h^2(t_2^2+1) \cr \vdots \cr \vdots \cr h^2(t_{n-2}^2+1) \cr h^2(t_{n-1}^2+1)-5(1-3h)\end{pmatrix}.$$}
\par \medskip \noindent 
%
Question 1.F: Use Matlab to solve the problems in 1.D and 1.E using the step sizes of 0.1 and 0.05. Can shooting method be used to solve either 1.D or 1.E?
\par \medskip \noindent 
%
Question 1.G: Formulate the matrix solution to 1.E using centered difference for the second derivative and backward difference for the first derivative. 
\par \noindent
{\color{teal}$$ \begin{pmatrix} (6h^2-3h-2) & 1 &  &   0 \cr (1+3h) & (6h^2-3h-2) & 1 & & & \cr  \ \ \ \ \ \  \ddots & \ddots & \ \ \ \ \ \ \ddots &  & \cr  \cr & (1+3h) & (6h^2-3h-2) &  1 \cr 0 &  & (1+3h) & (6h^2-3h-2) \end{pmatrix} \begin{pmatrix} y_1 \cr y_2 \cr \vdots \cr \vdots \cr y_{n-2} \cr y_{n-1}\end{pmatrix} $$
$$= \begin{pmatrix} h^2 (t_1^2+1)-3(1+3h) \cr h^2(t_2^2+1) \cr \vdots \cr \vdots \cr h^2(t_{n-2}^2+1) \cr h^2(t_{n-1}^2+1)-5\end{pmatrix}.$$}
\par \bigskip \par
{\bf Section 2 Linear Systems}  \\
You can find reference to solving linear systems in Chapter 4 of \cite{KK09} \\
%
Question 2.A:  Describe the differences between Gaussian Elimination with Back Substitution versus the Gauss-Jordan Method. 
\par \medskip \noindent 
%
Question 2.B:  What are the possible outcomes to solving a linear system $\ds A\vec x = \vec b$, please be as complete as possible in your answer.
\par \medskip \noindent 
%
Question 2.C: Describe all possible solutions using Gaussian Elimination to:
\begin{align*}
    4y + z &= 20 \cr 
    2x - 2y + z &= 0 \cr
    x + z &= 5
\end{align*}
(If no solutions exist then please state that outcome.)
\par \medskip \noindent 
%
Question 2.D:  Describe all possible solutions using Gaussian Elimination to:
\begin{align*}
    2x + z + w &= 5 \cr 
    y - w&= -1 \cr
    3x - z - w &= 0\cr
    4x + y + 2z + w &=9
\end{align*}
{\color{teal}$x=1$, $y=t-1$, $z=3-t$. and $w=t$}
\par \medskip \noindent 
%
Question 2.E:  Describe all possible solutions using Gaussian Elimination to:
\begin{align*}
    -3x - 3y &= 1 \cr 
      x + y &= -1
\end{align*}
{\color{teal} No Solution}
\par \medskip \noindent 
%
Question 2.F:  Describe all possible solutions using Gaussian Elimination to:
\begin{align*}
    x + z + w &= 3 \cr 
    y + w&= -1 \cr
    2x - y +2z + w &= 7\cr
    x + 2y + z + 3w &=1
\end{align*}
{\color{teal}$x=3-t-u$, $y=-1-t$, $z=u$. and $w=t$}
\par \medskip \noindent 
%
Question 2.G: Describe all possible solutions using Gaussian Elimination to:
\begin{align*}
     y - 2w &= 1 \cr 
    x + y + z  &= 4 \cr
    x + 3y + z  -4w &= 6\cr
    x + 2y + z - 2w &= 3
\end{align*}
{\color{teal}No Solution} 
\par \medskip \noindent 
%
Question 2.H: Describe all possible solutions using Gaussian Elimination to:
\begin{align*}
     x + y - 2z &= -1 \cr 
    x - y -3z  &= 0 \cr
    x - 3y -5 z  &= 0 
\end{align*}
(If no solutions exist then please state that outcome.) {\color{teal} solution is x=2, y=-1, and z=1}
\par \medskip \noindent 
%
Question 2.I: Describe all possible solutions using Gaussian Elimination to:
\begin{align*}
     x + y - 2z &= 3 \cr 
    x - y -3z  &= 1 \cr
    x - 3y -4 z  &= -1 
\end{align*}
(If no solutions exist then please state that outcome.) {\color{teal} solution is x=2 +5/2 t, y=1-(t/2), and z=t}
\par \medskip \noindent 
%
Question 2.J: Describe all possible solutions using Gaussian Elimination to:
\begin{align*}
     x + y - 2z &= -3 \cr 
    x - y -3z  &= 1 \cr
    x - 3y -4 z  &= 0 
\end{align*}
(If no solutions exist then please state that outcome.) {\color{teal} no solution}
\par \medskip \noindent 
%
Question 2.K: Describe all possible solutions using Gaussian Elimination to:
\begin{align*}
     x + 3z &= -2 \cr 
    -5 x + 2y - z  &= 0 \cr
    3x - y + 6z  &= -1 
\end{align*}
(If no solutions exist then please state that outcome.) {\color{teal} solution is x=1, y=2, and z= -1}
\par \medskip \noindent 
Question 2.L: Describe all possible solutions using Gaussian Elimination to:
\begin{align*}
     2x + y = 4 \cr 
    2 x + y - z  &= -1 \cr
    y + z  &= 8 
\end{align*}
(If no solutions exist then please state that outcome.) {\color{teal} solution is x=1/2, y=3, and z= 5}
\par \bigskip \noindent 
%
{\bf Section 3 Numerical Techniques for Solving Linear Systems}\\
\\
%
Question 3.A:  Describe what is meant with Gaussian Elimination with partial pivoting and discuss why partial pivoting would be necessary. 
\par \medskip \noindent 
%
Question 3.B:  Define the condition number for a matrix and explain what the implications of a condition number might mean for solving a linear system.  Is it desirable to have a small or large condition number?\par
{\color{teal} The condition number can be estimated by taking the ratio of the magnitude of the largest eigenvalue (largest in magnitude) for a matrix over the magnitude of the smallest eigenvalue (smallest in magnitude) for a matrix. Thus, a singular matrix would have an infinite condition number since a singular matrix has an eigenvalue equal to zero, officially we should say the condition number is not defined for singular matrices. For a nonsingular matrix, the condition number is always positive and it is always greater than, or equal to, one. The smaller the condition number is the better conditioned it is and the better behaved numerical methods are for solving a well conditioned system. If the condition number is large then it is known as ill-conditioned and these systems can present challenges to numerical solvers. }
\par \medskip \noindent 
%
Question 3.C: What are the advantages to using the Matlab function of A\textbackslash b to solve a linear system instead of using $\ds A^{-1}b$? Please be as complete as possible in your answer. Are there any circumstances where it is more advantageous to use $\ds A^{-1}$?\par
{\color{teal} The function of A\textbackslash b includes computational procedures that make the calculation more stable and more accurate. The use of pivoting would be a simplistic analogy for the kinds of things A\textbackslash b does to help make the solution more stable and more accurate. In addition, A\textbackslash b includes elements to speed up the calculations since calculating the full inverse of a matrix is a very expensive to accomplish computationally. }
\par \medskip \noindent 
%
Question 3.D:  Use three iterations of Gauss-Seidel with initial guess of [1; 2; 3;] for the system:
\begin{align*}
    12x + 7y + 3z &= 17 \cr 
    3x + 6y + 2z  &= 9 \cr
    2x + 7y - 11z &= 49
\end{align*}
Use Matlab to solve this system and compute the absolute relative error between the Gauss-Seidel approximation and the exact solution. 
\par \medskip \noindent 
%
Question 3.E: Give the definition of a singular linear system and a nonsingular system. Please describe how one can identify if a system is singular or nonsingular. 
%
\par \bigskip \noindent 
{\bf Section 4 Extra Questions}\\
\par \medskip \noindent 
Question 4.A: State the definition of local truncation error for solving a differential system
$$\frac {d\vec w}{dt} = \vec F(t,\vec w), \ \ \vec w(0)=\vec w_0.$$
{\color{teal}
$$LTE=\frac {\Vert \vec w(t_1)-\vec w_1 \Vert}{\Delta t}$$}
\par \medskip \noindent 
Question 4.B: State the definition for order of convergence for a numerical method solving a system. In this case, the output of the method involves vectors, $\ds \vec w_k$ converging to a solution $\ds \vec w$, i.e., $\ds \vec w_k \rightarrow \vec w$ as $k\rightarrow \infty$.
\par \medskip \noindent 
Question 4.C: Explain the link between finding eigenvalues and the root finding methods that were covered early in the course. \par
{\color{teal} The explanation is really in the definition. Eigenvalues are roots to the characteristic equation, i.e. eigenvalues are solutions to the equation $\ds \det (A - \lambda I) = 0$. }
\par \medskip \noindent 
Question 4.D: Use Matlab to compute the condition number for the system:
\begin{align*}
    x + y + z &= 2\cr 
    x + 2z  &= 1 \cr
    2x + y + 3.0001z &= 3.0001
\end{align*}
Interpret what the condition number might mean for solving this system. 
% Add references here, list alphabetically according to last name of primary author.

\section*{References}
\beginrefs

\bibentry{KK09} {\sc Autar Kaw} and {\sc E. Eric Kalu}, {\it Numerical Methods with Applications (2nd ed.)}, 2009. Website: \href{http://autarkaw.com/books/numericalmethods/index.html}{book website} .

\bibentry{Matlab} {\sc Matlab website}, \href{https://www.mathworks.com}{www.mathworks.com}, August 2019. % put exact and full link in the first listing right after href

\endrefs
\end{document}