%Every LaTeX file needs a documentclass declaration.
%Possibilities are article, book, letter.  Font size is also declared.

\documentclass[10pt]{article}

%special packages used for symbols, formatting, etc.

\usepackage{amsmath} % contains the align* environment, which is great for manipulating formulas
\usepackage{amssymb} % contains common symbols
\usepackage{amsthm} % has the proof environment
\usepackage[margin=1in]{geometry} % specifies page properties, such as the margin
%\usepackage{siunitx} % useful for typesetting units
\usepackage{xcolor}

% User-defined commands

\newcommand{\newprob}{\medskip \hrule \medskip}
\newcommand{\fanc}[1]{\mathbb{#1}}
\newcommand{\rn}[1]{\fanc{R}^{#1}}
\def\qed{\hspace*{\fill}\rule{1.854mm}{3mm}}  % the fancy box at the end of a proof

%%%%%%%%%%%%%%%%%%%%%%%%%%%%%%%%%%%%%%%%%%%%%%

%beginning of document, every \begin{} also requires an \end{} command.

%\renewcommand{\baselinestretch}{2}

\begin{document}

\pagestyle{empty}  %suppress page numbers, etc.

\begin{center}  %center command, also see flushright, flushleft

{\bf MATH 423-01  Advanced Calculus I

Homework \#2

Assigned: September 7, 2022

Due: September 14, 2022}

\textbf{\textcolor{red}{\underline{To Check}: 5abcd}}

\end{center}

\medskip

\hrule   %horizontal line

\bigskip

% list environment: description, itemize, and enumerate

\begin{enumerate}

%%%%%%%%%%%%%%%%%%%%%%

\item ~[Aitchison, A.] Describe each of the following sets as the empty set, $\mathbb{R}$, or in interval notation, as appropriate.

	\begin{enumerate}
	
	\item  $$\bigcup_{n=1}^{\infty} \left( -n, n \right)$$
\par \medskip
	\textbf{\textcolor{black}{\underline{Answer}:}}
\textcolor{black}{In this case, there is an $x$ that exists within $\mathbb{R}$ such that there exists a value called $N$.  So, it can be written at $N \leq x < N+1$.  Therefore, if $x \geq 0$, then $x \in (-N-1, N+1)$ and if $x < 0$, then $x \in (N-1, 1-N)$.  Therefore, $\cup(-n,n) = \mathbb{R}$.}
	
	\item  $$\bigcap_{n=1}^{\infty} \left( -\frac{1}{n}, \frac{1}{n} \right)$$
	\par \medskip
	\textbf{\textcolor{black}{\underline{Answer}:}}
\textcolor{black}{In this set, it states that $-\frac{1}{n} < 0 < \frac{1}{n}, \forall n$.  Then, $0 \in \cap(-\frac{1}{n}, \frac{1}{n})$.  If $x > 0, \exists N$ such that $x > \frac{1}{N}$.  Therefore, $x \notin (-\frac{1}{N}, \frac{1}{N})$ meaning that $x \notin \cap(-\frac{1}{n},\frac{1}{n})$.  Similarly, if $x < 0$, then $x \notin \cup(-\frac{1}{n},\frac{1}{n})$, therefore $\cap(-\frac{1}{n},\frac{1}{n})={0}=[0,0]$.}

	\item  $$\bigcap_{n=1}^{\infty} \left( -\frac{1}{n}, 1 + \frac{1}{n} \right)$$
	\par \medskip
	\textbf{\textcolor{black}{\underline{Answer}:}}
\textcolor{black}{This set can be viewed similarly to (b) such that $\cap(-\frac{1}{n},1+\frac{1}{n}) = {0 \leq x \leq 1} = [0,1]$.}
	\item  $$\bigcup_{n=1}^{\infty} \left( -\frac{1}{n}, 2 + \frac{1}{n} \right)$$
	\par \medskip
	\textbf{\textcolor{black}{\underline{Answer}:}}
\textcolor{black}{Since $(-\frac{1}{n}, 2 + \frac{1}{n}) \subset (-1,3) \forall n$, then $\cup(-\frac{1}{n}, 2 + \frac{1}{n}) = (-1,3)$.}
	\end{enumerate}
	
\item  Prove the following statements using induction.  The first statement should be straightfoward.  The second statement is trickier to prove.  Make sure that can you prove (a) before attempting (b).

	\begin{enumerate}
	
	\item  ~[Griffith, B.] The quantity $n^3-n$ is divisible by $3$, for $n \in \mathbb{N}$.
 \begin{proof}
 Using induction, let $n =1$ be our starting value.  Plugging it into the equation $n^3-n$, we get $1^3-1=0$, which is divisible by 3.  Since this is true, then we can assume the same statement is true for the value $n = k$.  We have to prove that $n=k+1$, is true.  By plugging in $k+1$ into the equation we get the following algebraic breakdown:
 \begin{center}
 $n^3-n = k^3-k$
 \par \medskip
 $(k+1)^3-(k+1)$
 \par \medskip
 $k^3+3k^2+3k+1-k-1$
 \par \medskip
 $(k^3-k)+3(k^2-k)$
 \par \medskip
 $(k^3-k)+3k(k+1)$
 \end{center}
 This final equation is $n^3-n$ plus a multiple of 3.  Since we assumed that $n^3-n$ was a multiple of 3 with the $n=k$ feature, it follows that $(n+1)^3-(n-1)$ is also a multiple of 3.  So, since the statement "$n^3-n$ is divisible by $3$" is true for $n=1$, and it's truth for $k$ implies its truth for $k+1$, hence proving that the statement is true for all $n \in \mathbb{R}$.
 \end{proof}
	
	\item  ~[James, J.] The quantity $n^3+5n$ is divisible by $6$, for $n \in \mathbb{N}$.
 \begin{proof}
Since $1^3 + 5 * 1 = 6$, and 6 is obviously divisible by 6, the base case $(n = 1)$ of the induction holds. Now suppose that for a certain positive integer $n$, it is known that $n^3 + 5n$ is divisible by 6. We need to deduce that $(n + 1)^3 + 5(n + 1)$ is divisible by $6$. Expanding this quantity, collecting terms, and factoring, we find that:
\begin{equation}
    (n + 1)^3 + 5(n + 1) = (n^3 + 5n)+3n(n + 1) + 6.
\end{equation}
By the induction hypothesis, $(n^3 + 5n)$ is divisible by 6. On the other hand,
$3n(n+1)$ is obviously divisible by 3, and since one of the consecutive integers
$n$ and $n + 1$ is even, $3n(n + 1)$ is also divisible by 2. Thus all three terms
on the right-hand side of (1) are divisible by 6, and so the left-hand side is
divisible by 6. This completes the induction step of the proof.  Therefore, the quantity $n^3+5n$ is divisible by $6$, for $n \in \mathbb{N}$.
 \end{proof}	
	\end{enumerate}

\item  Let $y_1=1$, and for each $n \in \mathbb{N}$ define $y_{n+1} = (3y_n+4)/4$.

	\begin{enumerate}
	
	\item  ~[Delfosse, D.] Use induction to prove that the sequence satisfies $y_n < 4$, for all $n \in \mathbb{N}$.
 \begin{proof}
 By induction, we start with the base casae.  Since we are given that $y_1 = 1$ in the problem, then it's clear that $y_1 < 4$.  For the inductive step, we must assume that $y_k < 4$.  Then we must prove that $y_{k+1} < 4$.  We first plug in the $k+1$ into the original equation:
 \begin{center}
 $y_{k+1} = \frac{3y_k+4}{4}$
 
 $\frac{3}{4}y_k + 1$
 \end{center}
 Since this is the inductive hypothesis, $y_k < 4$, so we can plug in 4 into the new equation to get:
 \begin{center}
 $y_{k+1} = \frac{3}{4}y_k + 1 < \frac{3}{4}*4+1=3+1=4$
 \end{center}
 so, indeed $y_{k+1} < 4$.  Therefore, since both the base case and the inductive step are true, we can conclude by induction that $y_n < 4 \forall n \in \mathbb{N}$.
 \end{proof}
	
	\item  ~[Hale, A.] Use induction to prove that the sequence $(y_1, y_2, y_3, \ldots)$ is increasing.  A sequence is \emph{increasing} if whenever $m < n$ it must be that $y_m \leq y_n$.  Hint: focus on $n$ and $n+1$, rather than a generic $m$ and $n$.
 \begin{proof}
 In order to prove that  $(y_1, y_2, y_3, \ldots)$ is increasing, we must look at the equation that $y_n < y_{n+1} \forall n \in \mathbb{N}$.  By induction, we can prove the base case.  We must prove that $y_1 < y_2$; since $y_1 = 1$, then: 
 \begin{center}
     $y_2 = \frac{3y_1+4}{4} = \frac{7}{4}$.
 \end{center}
 Since $1 < \frac{7}{4}$, then the statement $y_1 < y_2$ is true.  For the inductive step, assume that $y_k < y_{k+1}$.  We must prove that $y_{k+1}<y_{k+2}$.  Since we have $y_k < y_{k+1}$, we have that: 
 \begin{center}
  $y_{k+1} = \frac{3y_k+4}{4} < \frac{3y_{k+1}+4}{4} = y_{k+2}$    
 \end{center}
 where the equalities each follow from the definition of the $y_i$'s.
 Therefore, we conclude by induction that $y_n < y_{n+1} \forall n \in \mathbb{N}$.
 \end{proof}	
	\end{enumerate}
	
\item  Use the Triangle Inequality to prove the following inequalities.

	\begin{enumerate}
	
	\item  ~[Harter, J.] $|a-b| \leq |a| + |b|$.  Hint: hide the negative sign inside a substitution.
 \begin{proof}
 To begin this proof, we must define the Triangle Inequality.  It states that by letting $x$ and $y$ be real numbers in $\mathbb{R}$, then $|x + y| \leq |x| + |y|$.  In this case, we are proving the Triangle Inequality but with $a$ and $b$.  Let $x = a$ and $y = -b$.  Then our restated triangle inequality is $|a + (-b)| \leq |a| + |(-b)|$.  However, $|a+(-b)| = |a-b|$ and $|-b|=|-1*b|=|-1|*|b|=|b|$, so we have proven that $|a-b| \leq |a| + |b|$.
 \end{proof}	
	\item  ~[Kline, L.] $||a| - |b|| \leq |a-b|$.  Hint: split this into two cases, (i) $|a| - |b| \leq |a-b|$ and (ii) $-(|a| - |b|) \leq |a-b|$.  For (i) try to prove that $|a| \leq |b| + |a-b|$.
 \begin{proof}
In this proof, we must split it into two cases: 
\begin{center}
 $(i) |a|-|b| \leq |a-b|$
 
 $(ii) -(|a|-|b|) \leq |a-b|$
\end{center}
For case (i), let $m = a-b$ and $n = b$.  By the Triangle Inequality theorem, $|m+n| \leq |m| + |n|$.  By substitution, it can be rewritten as $|a-b+b| \leq |a-b|+|b|$.  Simplifying it down more, and subtracting $|y|$ from both sides, we get that $|a|-|b| \leq |a-b|$ proving case (i).

For case (ii), a similar procedure can be done.  Let $m = b-a$ and $n = a$.  By rewriting the case and Triangle Inequality theorem, $|m+n| \leq |m| + |n|$.  By substitution, $|b-a+a| \leq |b-a|+|a|$, which simplifies to $|b| \leq |b-a| + |a|$.  By subtracting $|b|$ and $|b-a|$ from each side, then it can be written as $|a| - |b| \geq -|b-a|$.  By order of subtraction, then $|a| - |b| \geq -|a-b|$, thus proving case (ii).

Therefore, by the fact that if $|a| \leq b$, then $-b \leq a \leq b$, then $||a| - |b|| \leq |a-b|$.
 \end{proof}	
	\end{enumerate}
	
\item  ~[Lauen, A.] Compute the supremum and infimum for each of the following sets.  You don't need a rigorous proof, but provide details.  Hint: write out the members of the set.  There may be an infinite number of elements, but write out enough of them to get a sense for the elements of the set.

	\begin{enumerate}
\item  $A_1 = \{n \in \mathbb{N}: n^2 < 10\}$.
  \par \medskip
	\textbf{\textcolor{black}{\underline{Answer}:}}
\textcolor{black}{$\sup{(A_1)}=3$ and $\inf{(A_1)} = 1$.}
\begin{proof}
  The reason that $\sup{(A_1)}=3$ is that 3 is an upperbound of the set $A_1 = \{n \in \mathbb{N}: n^2 < 10\}$.  In this case, the values included would be $\left\{1,2,3\right\}$.  Thus, $\sup{(A_1)} \leq 3$ but $3 \in A_1$.  Therefore $\sup{(A_1)}=3$.  A similar argument can be made for $\inf{(A_1)} = 1$, since it is the lowerbound located within the set.  
\end{proof}  

	
	\item  $A_2 = \left\{\frac{n}{m+n}: m, n \in \mathbb{N}\right\}$.
	\par \medskip
	\textbf{\textcolor{black}{\underline{Answer}:}}
\textcolor{black}{$\sup{(A_2)}=1$ and $\inf{(A_2)} = 0$.}
\begin{proof}
The reason why $\sup{(A_2)}=1$ is because it is the least upper bound in the set $A_2 = \left\{\frac{n}{m+n}: m, n \in \mathbb{N}\right\}$.  When writing out the elements, it goes:$\left\{\frac{1}{2}, \frac{1}{3}, \frac{1}{3}, \frac{1}{4},...,\frac{2}{3}, \frac{2}{4}, \frac{2}{5},...,\frac{3}{4}, \frac{3}{5}, ...\right\}$ Now for a fixed $m$ value, let 
\begin{center}
\[\lim_{n\to\infty} \frac{n}{m+n}\] 
\[\lim_{n\to\infty} \frac{1}{\frac{m}{n} + 1} = 1\]
\end{center}
Also, $\frac{n}{m+n} = \frac{n+m-m}{m+n}=1-\frac{m}{m+n}$.  Note that $\forall m, n \in \mathbb{N}, \frac{n}{m+n} \leq 1$, meaning $\sup{(A_2)} =1$.
The values of $n$ and $m$ as they are increased by 1 make the fraction become smaller and smaller towards zero but never quite hit zero.  This idea helps with making the $\inf{(A_2)} = 0$.  
\end{proof}
	\item  $A_3 = \left\{\frac{n}{2n+1}: n \in \mathbb{N}\right\}$.
		\par \medskip
	\textbf{\textcolor{black}{\underline{Answer}:}}
\textcolor{black}{$\sup{(A_3)}=\frac{1}{2}$ and $\inf{(A_3)} = \frac{1}{3}$.}
\begin{proof}
The reason why $\sup{(A_3)}=\frac{1}{2}$ and $\inf{(A_3)} = \frac{1}{3}$ is when writing out the set elements, it goes $\left\{\frac{1}{3}, \frac{2}{5}, \frac{3}{7}, \frac{4}{9},...\right\}$.  As it can be seen, as the values of $n$ go up, the values of the set get smaller and smaller.  To prove this, we can say that $2n+n \geq 2n+1, \forall n \in \mathbb{N}$.  Simplifying it down, it becomes $3n \geq 2n+1$.  Simplifying it down more, then $n \geq \frac{2n}{3} + \frac{1}{3}$.  This can be rewritten as $\frac{n}{2n+1} \geq \frac{1}{3}, \forall n \in \mathbb{N}$.  Since $\frac{1}{3} \in A_3$ and is the smallest element in $A_3$, then $\inf{(A_3)} = \frac{1}{3}$.  A similar route can be taken for the supremum by stating that $2n < 2n+1, \forall n \in \mathbb{N}$.  Simplifying this equation, we get that $\frac{n}{2n+1} < \frac{1}{2}, \forall n \in \mathbb{N}$.  Since this is the greatest lower bound, it can be concluded that $\sup{(A_3)} = \frac{1}{2}$.
\end{proof}
	\item  $A_4 = \left\{ \frac{m}{n} : m, n \in \mathbb{N} \textnormal{ and } m+n \leq 10\right\}$.
		\par \medskip
	\textbf{\textcolor{black}{\underline{Answer}:}}
\textcolor{black}{$\sup{(A_4)}=9$ and $\inf{(A_4)} = \frac{1}{9}$.}
\begin{proof}
The reason why $\sup{(A_4)}=9$ and $\inf{(A_4)} = \frac{1}{9}$ is when writing out the elements, one can know that $m$ and $n$ can be the same but cannot have either $m$ or $n$ be $0$.  The elements are as follows: 

$\left\{\frac{1}{1}, \frac{1}{2}, \frac{1}{3},\frac{1}{4},\frac{1}{5},\frac{1}{6},\frac{1}{7},\frac{1}{8},\frac{1}{9},..., \frac{2}{1}, \frac{2}{2},\frac{2}{3},...,\frac{2}{8}, ..., \frac{3}{1},\frac{3}{2},...,\frac{3}{7},..., \frac{4}{1},\frac{4}{2},...,\frac{4}{6},\frac{5}{1},\frac{5}{2},...,\frac{5}{5},\frac{6}{1},...,\frac{6}{4},\frac{7}{1},\frac{7}{2},\frac{7}{3},\frac{8}{1},\frac{8}{2}, \frac{9}{1}\right\}$

In the case of this large set, $\forall n \in A_4, n \leq 9$ and $9 \in A_4$, so $\sup{(A_4)} = 9$ and $\forall n \in A_4, n \geq \frac{1}{9}$ and $\frac{1}{9} \in A_4$, so $\inf{(A_4)} = \frac{1}{9}$.
\end{proof}
	\end{enumerate}
	
\newpage

\item  ~[Nupen, R.] Prove that if $a$ is an upper bound for $A$ and $a$ is also an element of $A$, then it must be that $a = \sup{(A)}$.  Hint: prove that $a \leq \sup{(A)}$ and $\sup{(A)} \leq a$.
 \begin{proof}
 By contradiction, suppose that $a \neq sup(A)$.  Since $a$ is an upper bound for $A$, it follows taht it is larger than the smallest upper bound: $a > sup(A)$.  However, since $sup(A)$ is an upper bound for $A$, it is larger or equal to all elements of $a$; in particular $a \in A$, so $sup(A) \leq a$, a contradiction with our previous result that $a > sup(A)$.  Hence, our initial assimption was false and $a = sup(A)$.
 \end{proof}	
 
\item  ~[Everyone, Lindskov, I.] Assume that $A$ and $B$ are nonempty, bounded above, and that $A \subseteq B$.  Prove that $\sup{(A)} \leq \sup{(B)}$.
\begin{proof}
Let $b = sup(B)$ and $a = sup(A)$.  Then $b \geq x, \forall x \in B$ and $b \geq x, \forall x \in A$.  This means that $b$ is an upper bound of $A$.  But $a$ is the least upper bound of $A$, hence $a \leq b$.  Substituting in the values, it becomes $sup(A) \leq sup(B)$. 
\end{proof}


\item ~[Everyone, Krason, T.] 

	\begin{enumerate}

	\item  Write a formal definition for the terms \emph{bounded below}, \emph{lower bound}, and \emph{infimum}. 
\par \medskip
	\textbf{\textcolor{black}{\underline{Answer}:}}
\textcolor{black}{A set $A \subseteq \mathbb{R}$ is \emph{bounded below} if $\exists u \in \mathbb{R}$ such that $x \geq u, \forall x \in A$. We call $u$ a \emph{lower bound} of A.  A real number $s$ is the greatest lower bound \emph{(infimum)} of the set $A \subseteq \mathbb{R}$  if: }

\begin{center}
\par \medskip
1.	$s$ is a lower bound of $A$. 
\par \medskip
2.	If $u$ is any lower bound for $A$, then $s \geq u$. 
\par \medskip
\end{center}
\par \medskip
\textcolor{black}{We write the infimum as $s = inf(A)$.}

	\item  State and prove a version of our lemma concerning the supremum and $\varepsilon > 0$ for the infimum.
\begin{proof}
Our lemma states to assume $s \in \mathbb{R}$ is an upper bound for a set $A \subseteq \mathbb{R}$.  Then $s=sup(A)$ if and only if $\forall \epsilon>0 ,\exists a \in A$ such that $s-\epsilon<a$.  There are two directions to prove this lemma.  In this case, we will prove the forward case.  In the forward case, assume $s$ is an upper bound of $A$ and $s=sup(A)$.  This means that given any upper bound $u$, it must be that $s \leq u$.  Then let $\epsilon >0$.  Consider $s-\epsilon<s$ and $s-\epsilon$ is not an upper bound.  With these notions, it can be stated that $\exists a \in A$ such that $s-\epsilon<a$.  The backwards case can be proven as well using two separate cases that helps with determining that $s=sup(A)$.
\end{proof}
	\end{enumerate}
	
\item ~[Lewis, J.] 

	\begin{enumerate}
	
	\item  Let $A$ be bounded above, and define $B = \{b \in \mathbb{R}: b \textnormal{ is a lower bound for } A\}$.  Prove that $\sup{(B)} = \inf{(A)}$.  Hint: first prove that $\sup{(B)}$ exists.  Then prove that $\sup{(B)}$ satisfies the definition of $\inf{(A)}$.
	 \begin{proof}
Given that $A$ is bounded above, then $\exists y \in \mathbb{R}$ such that $\forall a \in A$, $|a| \leq y$.  Now if $A$ is unbounded below, then $A$ has no lower bound, meaning $B = \emptyset$.  This also means that if $A$ is bounded below, then $A$ has a lower bound, meaning $B = \emptyset$.  To define $B$, for any $x \in B$, $x$ is a lower bound of $A$, i.e.$x \leq a \leq y, \forall a \in A$, so $B$ is a bounded above set.  As $B$ is a bounded above subset of $\mathbb{R}$, then $B$ has a supremum, hence $sup(B)$ exists.  We must split this notion into two cases:
\begin{center}
(i) If $A$ is unbounded below

(ii) If $A$ is bounded below
\end{center}
For case (i), if $A$ is unbounded below, then $A$ has no lower bound, so $B = \emptyset$ and also $inf(A_ = \emptyset = sup(B)$.

For case (ii), if $A$ is bounded below, then $B$ is a non-empty set.  $B$ then contains all the lower bounds of A, so if $x \in B$, then $x \leq a, \forall a \in A$.  Clearly, $a$ is an upper bound of $B$, $\forall a \in A$ i.e. every element of $A$ is an upper bound of $B$.  As $A$ has an infimum, then $A$ is bounded below and a subset of $\mathbb{R}$, so $inf(A)$ exists.  Let $m = inf(A)$.  Then, $m$ is the greatest lower bound, so for any lower bound $y$ of $A$, $y \leq m \leq a, \forall a \in A$.  As $y$ is a lower bound, so $y \in B$. Therefore, $m$ is the least upper bound of $B$ meaning that $sup(B) = m$.

Therefore, $sup(B) = m = inf(A)$, which means that $sup(B) = inf(A)$.
 \end{proof}	
 
	\item  Use (a) to explain why there is no need to assert that an infimum exists as part of the Axiom of Completeness.
\par \medskip
	\textbf{\textcolor{black}{\underline{Answer}:}}
\textcolor{black}{The reason why there is no need to assert that an infimum exists as part of the Axiom of Completeness is because as soon as we consider that $sup(B)$ exists, then $B \neq \emptyset$, i.e. the lower bound of $A$ must exist, i.e. $A$ is bounded below.  $A$ is a non-empty subset of $\mathbb{R}$, which is bounded below, so from the completeness property of $\mathbb{R}$, we can say that $inf(A)$ exists but not as a part of the Axiom of Completeness. }
	\end{enumerate}

\end{enumerate}

\end{document}